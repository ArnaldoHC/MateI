\documentclass{article}
\usepackage[utf8]{inputenc}
\usepackage{amsmath}
\usepackage{amsfonts}
\usepackage{amssymb}
\usepackage{graphicx}
\usepackage{enumitem}

\title{Tarea 1}
\date{}

\begin{document}

\maketitle

\textbf{Valor:} 5 puntos \\
\textbf{Técnicas Cualitativa y Numérica} \\
\textbf{Fecha de entrega:} Viernes 28 de febrero \\
La entrega es en físico durante la clase. \\
No olvides poner énfasis en los desarrollos correspondientes. \\
Si usas algún lenguaje de programación para resolver los ejercicios, por favor adjunta el código o el diagrama de flujo.

\section*{1. Campos de direcciones}

\begin{enumerate}
    \item De la ecuación diferencial \( y' = x + \sin(y) \):
    \begin{enumerate}[label=\alph*)]
        \item Una curva solución pasa por el punto \( (1, \pi/2) \). ¿Cuál es su pendiente en ese punto?
        \item Justifica porque para \( x > 1 \) las soluciones son crecientes.
        \item Muestra que la concavidad de cada solución tiene por función \( 1 + x \cos(y) + \frac{1}{2} \sin(2y) \). Justifica cada uno de los pasos.
        \item Una curva solución pasa por el punto \( (0, 0) \). Demuestra que la curva tiene un mínimo relativo en \( (0, 0) \).
    \end{enumerate}
    
    \item Sea \( \phi(x) \) la solución del problema con valor inicial \( y' = x - y \), \( y(0) = 1 \). Verifica que \( y = x - 1 \) es una solución de \( y' = x - y \) y explica por qué la gráfica de \( \phi(x) \) siempre está por arriba de la recta \( y = x - 1 \).
    
    \item Traza las isoclinas y bosqueja varias curvas soluciones, incluyendo las curvas que satisfacen las condiciones iniciales dadas:
    \begin{enumerate}[label=\alph*)]
        \item \( y' = x + 2y \), \( y(0) = 1 \)
        \item \( y' = x^2 \), \( y(0) = 1 \)
    \end{enumerate}
    
    \item Identifica la ecuación con su campo correspondiente. Justifica tu elección.
    \begin{enumerate}[label=\roman*)]
        \item \( \frac{dy}{dx} = x - 1 \)
        \item \( \frac{dy}{dx} = 1 - y^2 \)
        \item \( \frac{dy}{dx} = y^2 - x^2 \)
        \item \( \frac{dy}{dx} = 1 - x \)
        \item \( \frac{dy}{dx} = 1 - y \)
        \item \( \frac{dy}{dx} = x^2 - y^2 \)
        \item \( \frac{dy}{dx} = 1 + y \)
        \item \( \frac{dy}{dx} = y^2 - 1 \)
    \end{enumerate}
\end{enumerate}

\section*{2. Línea Fase}

\begin{enumerate}
    \item[5.] De las siguientes cuatro funciones \( f(y) \), esboza la línea fase para la ecuación diferencial autónoma \( \frac{dy}{dt} = f(y) \).
    
    \item[6.] De las siguientes líneas fases provenientes de ecuaciones autónomas \( \frac{dy}{dt} = f(y) \), esboza la gráfica de la correspondiente función \( f(y) \) asumiendo que \( y = 0 \) es la mitad del segmento mostrado en cada caso.
\end{enumerate}

\section*{3. Método de Euler}

\begin{enumerate}
    \item[8.] Aplica el método de Euler sobre la siguiente ecuación diferencial con valor inicial y en el intervalo dado. Debes incluir: a) tabla y b) gráfica.
    \[
    \frac{dy}{dx} = y^2 - 4x, \quad y(0) = 0.5; \quad 0 \leq x \leq 2, \quad \Delta x = 0.25
    \]
    
    \item[9.] Utiliza el método de Euler para aproximar la solución del siguiente problema de valor inicial en el valor indicado, tomando los pasos 1, 2, 4 y 8:
    \[
    \frac{dx}{dt} = 1 + t \sin(tx), \quad x(0) = 0. \quad \text{En } t = 1
    \]
    
    \item[10.] La ley de Stefan de la radiación establece que la tasa de cambio de temperatura de un cuerpo a \( T(t) \) Kelvin en un medio a \( M(t) \) Kelvin es proporcional a \( M^4 - T^4 \). Es decir,
    \[
    \frac{dT}{dt} = K \left[ M^4(t) - T^4(t) \right],
    \]
    donde \( K = 2.9 \times 10^{-10} \, \text{min}^{-1} \) y \( M(t) = 293 \) Kelvin. Si \( T(0) = 360 \) Kelvin, usa el método de Euler con \( \Delta x = 3 \) min para aproximar la temperatura del cuerpo después de:
    \begin{enumerate}[label=\alph*)]
        \item 30 minutos
        \item 60 minutos
    \end{enumerate}
\end{enumerate}

\end{document}