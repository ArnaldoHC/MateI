\documentclass[12pt]{exam}
\usepackage[utf8]{inputenc}		% Caracteres latinos
\usepackage[spanish]{babel}		% Idioma español
\usepackage{geometry}			% Organizar el documento
\usepackage{graphicx}			% Incluir gráficos
\usepackage{makecell}			% Para personalizar las celdas de una tabla
\usepackage[nohdr]{mathexam}	% Añadimos el paquete mathexam (sin header)
\usepackage{amsmath}
\usepackage{amsfonts}
\usepackage{amssymb}
\usepackage{mathtools}
\usepackage{tikz,pgfplots}
\usepgfplotslibrary{polar}
\usepackage[shortlabels]{enumitem}
 \renewcommand{\baselinestretch}{1.5}
\usepackage{mathtools}
\usepackage{bm}
\usepackage{esvect}
\usepackage[fleqn]{mathtools}
\usepackage{relsize}
\usepackage{multirow}
\usepackage{multicol}
\usepackage[document]{ragged2e}
 \usepackage{textpos}
\usepackage{tcolorbox}
\usepackage{hyperref}
\usepackage{mathdesign}

%\usepackage[]{mathptmx}        % A free version o Times Roman with mathematical symbols
%\usepackage{pzc}               % fuente cursiva (conjuntos) Zapf Chancery
%\usepackage{showframe}
%\usepackage{lipsum}

% DOCUMENTACIÓN DE LA CLASE EXAM
% http://ftp.inf.utfsm.cl/pub/tex-archive/macros/latex/contrib/exam/examdoc.pdf
% DOCUMENTACIÓN DE LA CLASE MATHEXAM
% http://ctan.dcc.uchile.cl/macros/latex/contrib/mathexam/doc/mathexam.pdf

% Definimos la geometría de la primera página
\geometry{
	a4paper,                    % Tamaño del documento
	hmargin = {1.7cm, 1.6cm}, 	% Margen horizontal izquierdo, derecho
	vmargin = {1cm, 1cm},	    % Margen vertical superior, inferior
	headsep = 4mm,				% Separación entre el encabezado y el texto
	head = .2cm,				% Tamaño del encabezado
	% marginparsep = 5mm, 		% Seperación entre las notas y el texto
	% marginpar = 1.5cm,		% Tamaño de las notas
	includeall,                 % incluye el encabezado, footer y notas dentro del tamaño del documento
	nomarginpar,	            % Elimina las notas
	foot = 1cm,                 % Tamaño del footer
	twoside,                	% Habilita el modo de impresión a doble cara
}

\selectlanguage{spanish}        % Selecciona el idioma
\spanishdecimal{.}

%\pagestyle{headandfoot}         % Nuestro examen tendrá encabezado y pié

% DEFINIMOS EL ENCABEZADO
%\header{
%\begin{tabular}{l c c c l}
%            \makecell{\includegraphics[height=2.5cm]{logo.png}} &
%            \makecell{\textbf{IPEA 215} \\Raúl Scalabrini Ortiz} &
%            \makecell{Examen} &
%            \makecell{Curso\\1er Año} &
%             \makecell[l]{Apellido y %Nombre:\enspace\makebox[2in]{\hrulefill}\\Fecha: \today}
%        \end{tabular}}{}{}

% DEFINIMOS EL PIE
%\rfoot{Página \thepage\ de \numpages}
\newcommand{\iuni}{\pmb{\hat{\imath}}}
\newcommand{\juni}{\pmb{\hat{\jmath}}}
\newcommand{\kuni}{\pmb{\hat{k}}}
\renewcommand{\sin}{\,\text{sen}\,}
\newcommand*\colvec[3][]{
    \begin{pmatrix}\ifx\relax#1\relax\else#1\\\fi#2\\#3\end{pmatrix}
}
%\colvec{a}{b} para dos 
% \colvec[a]{b}{c} para tres
% DOCUMENTO
\begin{document}

\centering


\Large 
\textbf{\huge Tarea 8 \\ \large Transformada de Laplace. Ecuaciones Diferenciales Parciales}

\small
Fecha de entrega Domingo 23 de Enero
\vskip10pt
% \flushleft
% \begin{tcolorbox}
% LEE ANTES DE COMENZAR LA TAREA
% \begin{enumerate}
%     \item Se califica sobre 10 por lo que es posible obtener 13 si se realizan también los ejercicios opcionales.
%     \item El ejercicio opcional puede sustituir a uno y solo uno de los ejercicio del bloque previo.
%     \item Subir el archivo en \textbf{PDF} a la plataforma con las páginas numeradas.

% \end{enumerate}
% \end{tcolorbox}
\normalsize

\pointpoints{punto}{puntos}
\pointformat{\bfseries\boldmath(\thepoints)}
\vskip10pt

    
    \begin{questions}
     % 1 % 
     % Nagle 7.5 e2,6,8,12
     \question
     Resuelve el problema de valores iniciales mediante el método de transformada de Laplace.
     \begin{enumerate}[a)]
         \item $y''-y'-2y=0$; $y(0)=-2$, $y'(0)=5$
         \item $y''-4y'+5y=4e^{3t}$; $y(0)=2$, $y'(0)=7$
         \item $y''+4y=4t^2-4t+10$; $y(0)=0$, $y'(0)=3$
         \item $w''-2w'+w=6t-2$; $w(-1)=3$, $w'(-1)=7$
     \end{enumerate}


     % 2 
     % Nagle 7.5 e27
     \question% 
     Resuelve el problema de tercer orden $y'''+3y''+3y'+y=0$ dados los valores iniciales $y(0)=-4$, $y'(0)=4$ y $y''(0)=-2$ usando el método de la transformada de Laplace.

     
     % 3 
     %Nagle 7.7 e2,3
     \question% 
     Usa el teorema de convolución para obtener una fórmula para la solución del problema de valores iniciales dado, donde $g(t)$ es continua por partes en $[0,\infty)$ y de orden exponencial.
     \begin{enumerate}[a)]
         \item $y''+9y=g(t)$; $y(0)=1$, $y'(0)=0$
         \item $y''+4y'+5y=g(t)$; $y(0)=1$, $y'(0)=1$
     \end{enumerate}


     % 4 
     % Nagle 7.9 e1,5
     \question% 
     Usa el método de la transformada de Laplace para resolver el problema con valores iniciales.
     \begin{enumerate}[a)]
         \item $\left\lbrace\begin{array}{ll}
              x'=3x-2y;&x(0)=1  \\
              y'=3y-2x;&y(0)=1 
         \end{array}\right.$
         \item $\left\lbrace\begin{array}{ll}
              x'=y+\sen t;&x(0)=2  \\
              y'=x+2\cos t;&y(0)=0 
         \end{array}\right.$
     \end{enumerate}
  \vskip20pt
  
     % 5 
     % Nagle 10.2 e17,21
     \question%
     \begin{enumerate}[a)]
         \item Resuelve el problema de flujo de calor:
         
         $\begin{array}{ll}
              \frac{\partial u}{\partial t}(x,t)=\beta\frac{\partial^2u}{\partial x^2}(x,t),&0<x<L,\;t>0  \\
              u(0,t)=u(L,t)=0,&t>0\\
              u(x,0)=f(x),&0<x<L
         \end{array}$
         \vskip10pt
         Donde $\beta=3$, $L=\pi$ y $f(x)=\sen x-7\sen3x+\sen5x$
         \newpage
         \item Resuelve el problema de la cuerda vibrante:
         
         $\begin{array}{ll}
              \frac{\partial^2u}{\partial t^2}=\alpha^2\frac{\partial^2u}{\partial x^2},&0<x<L,\;t>0  \\
              u(0,t)=u(L,t)=0,&t\geq0\\
              u(x,0)=f(x),&0\leq x\leq L\\
              \frac{\partial u}{\partial t}(x,0)=g(x),&0\leq x\leq L\
         \end{array}$
         \vskip10pt
         Donde $\alpha=3$,$L=\pi$, $f(x)=6\sen2x+2\sen6x$ y $g(x)=11\sen9x-14\sen15x$
     \end{enumerate}
     

        \end{questions}

\end{document}
