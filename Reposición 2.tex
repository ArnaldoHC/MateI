\documentclass[12pt]{exam}
\usepackage[utf8]{inputenc}		% Caracteres latinos
\usepackage[spanish]{babel}		% Idioma español
\usepackage{geometry}			% Organizar el documento
\usepackage{graphicx}			% Incluir gráficos
\usepackage{makecell}			% Para personalizar las celdas de una tabla
\usepackage[nohdr]{mathexam}	% Añadimos el paquete mathexam (sin header)
\usepackage{amsmath}
\usepackage{amsfonts}
\usepackage{amssymb}
\usepackage{mathtools}
\usepackage{tikz,pgfplots}
\usepgfplotslibrary{polar}
\usepackage[shortlabels]{enumitem}
 \renewcommand{\baselinestretch}{1.5}
\usepackage{mathtools}
\usepackage{bm}
\usepackage{esvect}
\usepackage[fleqn]{mathtools}
\usepackage{relsize}
\usepackage{multirow}
\usepackage{multicol}
\usepackage[document]{ragged2e}
 \usepackage{textpos}
\usepackage{tcolorbox}
\usepackage{hyperref}
\usepackage{mathdesign}

%\usepackage[]{mathptmx}        % A free version o Times Roman with mathematical symbols
%\usepackage{pzc}               % fuente cursiva (conjuntos) Zapf Chancery
%\usepackage{showframe}
%\usepackage{lipsum}

% DOCUMENTACIÓN DE LA CLASE EXAM
% http://ftp.inf.utfsm.cl/pub/tex-archive/macros/latex/contrib/exam/examdoc.pdf
% DOCUMENTACIÓN DE LA CLASE MATHEXAM
% http://ctan.dcc.uchile.cl/macros/latex/contrib/mathexam/doc/mathexam.pdf

% Definimos la geometría de la primera página
\geometry{
	a4paper,                    % Tamaño del documento
	hmargin = {1.7cm, 1.6cm}, 	% Margen horizontal izquierdo, derecho
	vmargin = {1cm, 1cm},	    % Margen vertical superior, inferior
	headsep = 4mm,				% Separación entre el encabezado y el texto
	head = .2cm,				% Tamaño del encabezado
	% marginparsep = 5mm, 		% Seperación entre las notas y el texto
	% marginpar = 1.5cm,		% Tamaño de las notas
	includeall,                 % incluye el encabezado, footer y notas dentro del tamaño del documento
	nomarginpar,	            % Elimina las notas
	foot = 1cm,                 % Tamaño del footer
	twoside,                	% Habilita el modo de impresión a doble cara
}

\selectlanguage{spanish}        % Selecciona el idioma
\spanishdecimal{.}

%\pagestyle{headandfoot}         % Nuestro examen tendrá encabezado y pié

% DEFINIMOS EL ENCABEZADO
%\header{
%\begin{tabular}{l c c c l}
%            \makecell{\includegraphics[height=2.5cm]{logo.png}} &
%            \makecell{\textbf{IPEA 215} \\Raúl Scalabrini Ortiz} &
%            \makecell{Examen} &
%            \makecell{Curso\\1er Año} &
%             \makecell[l]{Apellido y %Nombre:\enspace\makebox[2in]{\hrulefill}\\Fecha: \today}
%        \end{tabular}}{}{}

% DEFINIMOS EL PIE
%\rfoot{Página \thepage\ de \numpages}
\newcommand{\iuni}{\pmb{\hat{\imath}}}
\newcommand{\juni}{\pmb{\hat{\jmath}}}
\newcommand{\kuni}{\pmb{\hat{k}}}
\renewcommand{\sin}{\,\text{sen}\,}
\newcommand*\colvec[3][]{
    \begin{pmatrix}\ifx\relax#1\relax\else#1\\\fi#2\\#3\end{pmatrix}
}
%\colvec{a}{b} para dos 
% \colvec[a]{b}{c} para tres
% DOCUMENTO
\begin{document}

\centering


\Large 
\textbf{\huge Reposición 2 \\ \large }

\small
% Fecha de entrega Viernes 28 de Enero
\vskip10pt
% \flushleft
% \begin{tcolorbox}
% LEE ANTES DE COMENZAR LA TAREA
% \begin{enumerate}
%     \item Se califica sobre 10 por lo que es posible obtener 13 si se realizan también los ejercicios opcionales.
%     \item El ejercicio opcional puede sustituir a uno y solo uno de los ejercicio del bloque previo.
%     \item Subir el archivo en \textbf{PDF} a la plataforma con las páginas numeradas.

% \end{enumerate}
% \end{tcolorbox}
\normalsize

\pointpoints{punto}{puntos}
\pointformat{\bfseries\boldmath(\thepoints)}
\vskip10pt

    
    \begin{questions}
     % Tarea 4 2 % Nagle S4.1 E6
     \question% 
     Se aplica una fuerza externa $F(t)=2\cos2t$ a un sistema masa-resorte con $m=1$, $b=0$ y $k=4$, que está inicialmente en reposo; es decir, $y(0)=0$, $y'(0)=0$. Verifica que $y(t)=\frac{1}{2}t\sen2t$ da el movimiento de este resorte. ¿Qué ocurrirá a largo plazo (al aumentar $t$) con el resorte?

     % Tarea 4 8 % Nagle 4.3 e32,33
     \question%
     Un resorte en vibración sin amortiguamiento puede modelarse mediante el problema con valores iniciales $my''(t)+by'(t)+ky=0$ haciendo $b=0$.
     \begin{enumerate}[a)]
         \item Si $m=10\, kg$, $k=250\,kg/s^2$, $y(0)=0.3\,m$ y $y'(0)=-0.1\,m/s$, determina la función de movimiento para este resorte en vibración sin amortiguamiento.
         \item Cuando la ecuación de movimiento tiene la forma $y(t)=c_1e^{\alpha t}\cos\beta t+c_2e^{\alpha t}\beta t$, se dice que el movimiento es oscilatorio con frecuencia $\beta/2\pi$. Determina la frecuencia de oscilación para el sistema de resorte del inciso anterior.
         \item Agrega un término de amortiguamiento al mismo sistema haciendo $b=60\,kg/s$ y determina la función de movimiento y su frecuencia de oscilación.
         \item ¿Qué efecto tuvo el amortiguamiento sobre la frecuencia de oscilación y qué otros efectos tuvo sobre la solución?
     \end{enumerate}

     % Tarea 5 10 b % Nagle impreso 4.4 e3,5 Moya pp164 e k)
     \question%
    Mediante el método de reducción de orden, determina una segunda solución linealmente independiente a la solución dada $f(t)=e^t$ de $tx''-(t+1)x'+x=0$,  $t>0$.  
    % \begin{enumerate}[a)]
    %     \item $x^2y''-2xy'-4y=0$,  $x>0$;  $f(x)=x^{-1}$
    %     \item 
    %     \item $x^2y''-5xy'+8y=4x^2$;  $f(x)=x^2$
    % \end{enumerate}

     % Tarea 5 4 % Nagle S4.5 e18,20,22
     \question% 
     Determina la solución general de:
     \begin{enumerate}[a)]
    %  	\item	$y''-2y'-3y=3t^2-5$
        \item	$y''(\theta)+4y(\theta)=\sen\theta-\cos\theta$
        \item	$y''(x)+6y'(x)+10y(x)=10x^4+24x^3+2x^2-12x+18$
             %Tarea 5 8 % Nagle S4.6 e12,15
        \item $y''+y=3\sec t-t^2+1$
     \end{enumerate}

     %Tarea 4 7 % Nagle S4.2 e7,10,14 S4.3 e1,25
     \question%
     Resuelve las siguientes ecuaciones diferenciales.
     \begin{enumerate}[a)]
        %  \item $y'''+3y''-4y'-12y=0$
        %  b
         \item $2u''+7u'-4u=0$
        %  \item $z'''+2z''-4z'-8z=0$
        %  \item $y''+y´=0$ con $y(0)=2$ y $y'(0)=1$
        %  \item $y''-2y'+2y=0$ con $y(\pi)=e^{\pi}$ y $y'(\pi)=0$
        %  f
         \item $y'''-2y''-y'+2y=0$ con $y(0)=2$, $y'(0)=3$ y $y''(0)=5$
        %  \item $y^{iv}+13y''+36y=0$
        %  \item $4y''-4y'+y=0$
        %  i
         \item $y''+9y=0$
     \end{enumerate}

     %Tarea 5 5 % Nagle S4.5 29,30
     \question%
     Determina la solución del problema con valores iniciales $y''+2y'+y=t^2+1-e^t$;  $y(0)=0$,  $y'(0)=2$.
    %  \begin{enumerate}[a)]
    %     \item	$y''(\theta)-y(\theta)=\sen\theta-e^{2\theta}$;  $y(0)=1$,  $y'(0)=-1$
    %     \item	$y''+2y'+y=t^2+1-e^t$;  $y(0)=0$,  $y'(0)=2$
    %  \end{enumerate}
     
     %Tarea 5 7 % Nagle S4.6 e2,3,10
     \question%
     Determina una solución general usando el método de variación de parámetros de $y''+4y'+4y=e^{-2t}\ln t$.
%      \begin{enumerate}[a)]
%      	\item $y´´+y=\sec t$
%         \item $2x''-2x'-4x=2e^{3t}$
%         \item $y''+4y'+4y=e^{-2t}\ln t$
% \end{enumerate}

     % 4 % Maya pp148 e9(b,c,d),  Nagle impreso S4.3 e25(a,b,c)
     \question% 
     Determina si las soluciones dadas son linealmente independientes o dependientes en el intervalo.
     \begin{enumerate}[a)]
         \item $y_1(x)=xe^{2x}$, $y_2(x)=e^{2x}$; en el intervalo $(0,1)$
         \item $y_1(x)=x^2\cos(\ln|x|)$, $y_2(x)=x^2\sen(\ln|x|)$; en el intervalo $(0,1)$
        %  \item $y_1(x)=0$, $y_2(x)=e^x$; en el intervalo $(0,1)$
        %  \item $y_1(x)=1$, $y_2(x)=x$, $y_3(x)=x^2$; en el intervalo $(-\infty,\infty)$
         \item $y_1(x)=-3$, $y_2(x)=5\sen^2x$, $y_3(x)=\cos^2x$; en el intervalo $(-\infty,\infty)$
     \end{enumerate}



    \end{questions}

\end{document}
